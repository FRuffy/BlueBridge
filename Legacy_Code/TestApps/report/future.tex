%%%%%%%%%%%%%%%%%%%%%%%%%%%%%%%%%%%%%%%%%%%%%
\section{Future Work}
\label{sec:future}
%%%%%%%%%%%%%%%%%%%%%%%%%%%%%%%%%%%%%%%%%%%%%

\subsection{Implementing DSM}

BlueBridge serves as a proof of concept for DSM, and has shown the potential
merits of a DSM system for graph processing. Given the results shown in this
analysis, BlueBridge should be implemented on an actual cluster, leveraging
the potential of DSM. In order to do this, pages of memory in the DSM system
would contain both local pages of memory, and shared global pages of memory.
BlueBridge would load the external edges of the graph onto these shared global
pages, and synchronization of these pages between iterations would be done
by a shared state manager. One proposed implementation of this system is shown
in Figure \ref{fig:dsm}, in which outgoing external edge values are stored on
shared pages and requested by other machines for their remote pages.

\begin{figure}[h]
\includegraphics[width=\linewidth]{"fig/dsm_implementation"}
\caption{A proposed implementation of BlueBridge on distributed shared memory.}
\label{fig:dsm}
\end{figure}

\subsection{Offloading Work from Python}

The present state of the BlueBridge prototype comes with many performance
concerns. While Python provides a simple interface for the access of shared
memory, use of Python's \textit{Manager} library for access and control of
shared state was unacceptable from a performance perspective. \textit{cProfile},
the Python profiler, was used to investigate the performance of this library
and it was found that up to 80-90\% of the total runtime was spent in
\textit{Manager} library communication primitives, such as \textit{send()} and
\textit{recv()}. We believe that writing a custom shared state manager to handle
memory access in C would provide enormous performance benefits over the current
BlueBridge prototype.
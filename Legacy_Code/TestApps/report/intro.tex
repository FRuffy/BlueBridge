%%%%%%%%%%%%%%%%%%%%%%%%%%%%%%%%%%%%%%%%%%%%%
\section{Introduction}
\label{sec:intro}
%%%%%%%%%%%%%%%%%%%%%%%%%%%%%%%%%%%%%%%%%%%%%

Big data processing is complicated. Scientists and experienced
programmers alike struggle with managing and configuring clusters of
machines to processes large amounts of data. Frameworks like Hadoop
and Pregel~\cite{Hadoop09,Malewicz:2010:PSL:1807167.1807184} have
significantly eased the difficulty of big data processing, but they
remain intimidating for the layman. We noticed a trend in scientists
and researchers, finding that they wanted to \textit{"Just write
python code that ran on a bunch of computers"}. We investigated how to
make this dream a reality.

Running all Python code on clusters of machines would lead to sluggish
un-optimal code, due to immense network overhead.  Instead, we
concentrated our efforts on a common but difficult big data processing
task: graph processing. Many frameworks exist for distributed graph
processing~\cite{Malewicz:2010:PSL:1807167.1807184,Ching:2015:OTE:2824032.2824077,Kyrola:2012:GLG:2387880.2387884,Low:2012:DGF:2212351.2212354,Xin:2013:GRD:2484425.2484427,Gonzalez:2012:PDG:2387880.2387883},
and many general frameworks exist that are used for graph
processing~\cite{Vavilapalli:2013:AHY:2523616.2523633,Zaharia:2012:RDD:2228298.2228301,Isard:2007:DDD:1272996.1273005,Murray:2013:NTD:2517349.2522738}.
These frameworks vary in their complexity, but none are
\textit{"accessible"} for programmers that lack relevant experience.

By definition all programmers must be able to produce single threaded
programs. Some researchers have argued and demonstrated that single
threaded applications can outperform scalable distributed frameworks
with little effort~\cite{189908}, and that such applications should be
used as benchmarks for scalability. We evaluate these claims and
propose a simple framework for multi-threading optimal single threaded
applications.

With the exception of~\cite{Kyrola:2012:GLG:2387880.2387884}, the
aforementioned systems suffer a common pitfall to accessibility; they
expose the complexity of a distributed message passing system to the
user. Extensive work has been done to hide this complexity in the
abstraction of distributed shared memory (DSM)
~\cite{Keleher:1994:TDS:1267074.1267084,Power:2010:PBF:1924943.1924964,Morin:2004:KDP:1111682.1111729,Haddad:2001:MCL:374794.374800,Huang06vodca:view-oriented}.
The benefits of DSM have been ignored in recent years due to its flaws,
most notably fate sharing and sub optimal performance. Dismissing DSM may 
have been a shortsighted mistake. Ultra dense memory and the
approach of terabit-level bandwidth within a rack give modern racks the
appearance of a single machine, and has lead towards disaggregated
architectures~\cite{facebook-rack,machine,intel-rsa,seamicro,Han:2013:NSR:2535771.2535778}.
Such futuristic systems lend themselves naturally to DSM which
motivates our proposal for a corresponding computation framework.

The largest disadvantage of DSM is performance. Programmers can write
programs with terrible performance by failing to reason about the location
of memory, leading to memory thrashing. In computation where a high
degree of consistency between shared resources is needed, DSM is the 
wrong tool for the job. In contrast, when large amounts of computation can be
performed between memory synchronizations, DSM provides a simple and
efficient programming model. Graph processing suffers from a lack of
locality. In computations such as PageRank, a single iteration may
perform edge updates which require the synchronization of all
machines in a cluster. This problem can be largely avoided in practice
by carefully pre-processing graphs into equally-sized partitions, where 
the minimum number of graph edges exist across machines. The cost of pre-processing a
graph can be large; in some cases, the complexity of finding a good
partition is greater than solving the initial problem! Here we
demonstrate that the cost of graph partitioning is worth it for the
benefits that DSM provides.

In this paper we attack the problem of developing a simple and
efficient graph processing interface for DSM. Specifically we make the
following contributions.

\begin{itemize}
        \item An evaluation of single threaded applications as a baseline for scalability
        \item A simple framework for scaling single threaded applications
        \item A graph processing API for DSM
        \item A graph partitioning scheme optimal for DSM
        \item An evaluation of processing performance between partitioned DSM processing, and Pregel-style graph processing
\end{itemize}

The remainder of this paper is organized as follows. In
Section~\ref{sec:scalability} we discuss approaches and strategies for 
scalability. In Section~\ref{sec:autoscaling} we methodology for
automatically scaling up graph processing. Section~\ref{sec:related} 
describes other systems similar to ours, Section ~\ref{sec:future} discusses
future work, and Section~\ref{sec:conclusion} concludes the paper.

\section{Conclusion}
\label{sec:conclusion}

We have shown and discussed a new version of DSM which leverages in network
management to expose a NUMA machine to the user. This provides a simple
generic interface for the developer to program on, yet still provides the
benefits of distribution of data and compute. We describe the Camelot system,
which builds upon the BlueBridge system by adding multi-threading support,
different paging policies, and RAID for memory fault tolerance. We evaluated
each additions' performance impact and functionality. In our current setup, Camelot was able to scale 
up to eight cores, achieving a throughput of 3.5 Gbps with an average request latency of around 60 
microseconds. Application runtime can be greatly improved by implementing fast running 
replacement policies and the choice of policy has big impact on performance.
In memory RAID can significantly reduce
memory usage (over 50\% on common configurations) with a computational overhead
of \textasciitilde6\% compaired to competing solutions~
\cite{Ousterhout:2015:RSS:2818727.2806887}.
\section{Introduction} 
\label{sec:intro}

\ac{Outline is inline. Need to integrate rack-scale + faster interconnects
part.}

\ac{1. Data-intensive applications important large-scale computations -->
PageRank, etc.}

As data collected increases, the need for data-intensive large scale processing
increases. \todo{Statistic about Google/Facebook/Amazon processing}. Large scale
data processing is not limited to large technology companies, different
disciplines in the sciences have also taken advantage of computing power to make
advancements \todo{Cite some scientific use of HPC or DC}. 

\ac{2. Can run on single machine or can distribute --> depending on dataset,
	single machine will not work (i.e., too big for disk), distribution is best for
	commodity hardware but is hard to reason about distribution.}

The main dichotomy
which appears in large scale data process is single machine (high performance
computing) vs. clustered machines (distributed computing). Single machine
computing has an advantage in locality of data and compute, but has the
disadvantage of cost. Clustered computing on commodity hardware is cheaper, but
has the disadvantage of added complexity\footnote{This is not to imply that
programming HPC applications tend to be easier, they have their own set of
complexities, but they do not deal with complexities brought on by the
transference of data and compute over the network}. 

\ac{3. Previous work attempts to address that issue by abstracting the
	distributed nature of the cluster.}

A large amount of previous work attempts to reduce the complexity of
distributed processing by imposing abstractions~\cite{Dean2004,Low2010,Murray2013}.

\ac{4. But these previous works trade-off generality for performance.
	Essentially, only certain types of applications (classes) perform well. }

This, unfortunately, restricts the types of applications which can run on these
frameworks, often leading to developers attempting to force their workload into
one of these frameworks. 

\ac{5. Single machine implementations require programmer expertise to take
	advantage of optimizations (COST + HPC) which is not always guaranteed.}

Yet, in the single machine case, it has been shown that
many of these frameworks out performed by a single threaded laptop
implementation~\cite{COST}. McSherry showed that his implementation of certain
processing algorithms in RUST outperformed the most popular frameworks for that
algorithm. But, programming highly performant code requires knowledge of the
system (i.e., operating system tuning) and is not necessarily extensible 
(different tuning parameters per dataset and algorithm). The average programmer
needs to be able to reason about their algorithm but also not reason
about the underlying system to gain performance. 

\ac{6. We envision a framework which provides performance but also
	programmability by exposing a single rack-scale NUMA machine interface to a
	developer and handling all distribution of memory in the system.}

We propose a combined approach where the advantages of the single machine
implementation can be combined with the advantages of the distributed
implementation. We envision a system which provides a simple single threaded
interface for application developers to design and implement on. The underlying
system parallelizes and distributes that application as appropriate. 

\ac{7. To realize this system, we built Camelot, a network managed distributed
	shared memory system.}

We realize
the first part of this system in Camelot, the network managed distributed shared
memory backend. 

\ac{8. DSMs have been studied extensively, but never really adopted until
	recently due to performance concerns. With the onset of faster networks, more
	work has gone into improve the performance of DSMs and using them in processing
	frameworks. }

Distributed shared memory has be extensively studied in the past, but has
recently resurfaced as an promising area of research due to faster
interconnects. Traditionally, DSMs did not achieve adoption due to low
throughput and high latency of network data transfers as well as the added
complexity of fault tolerance. Recent work has attempted to address this issues
individually using different techniques~\cite{Ongaro2011,Nelson2015}.

\ac{9. For example, Grappa, does blah. We differ because we offload some
	management to the switch.}

Whereas Grappa attempts to solve the issue of performance
with \todo{blah}. 
We differ from both solutions as we attempt to improve
performance and fault tolerance by leveraging the network. 

\ac{10. We expand upon a previous system, BlueBridge, to bring it closer to
the overall system we envision by adding support for multi-threading, improved
paging policies, and fault tolerance.}

Camelot is the next generation of a previous system, BlueBridge, which
implemented the basis for network managed DSM. We expand upon BlueBridge to
bring it closer to our envisioned system by adding support for multi-threading,
improved paging policies, and fault tolerance.

% \begin{itemize}
%         \item processing data at scale is hard
%         \item scientists, not just computer scientists process data
%         \item python, and C are more understood / natural than spark/hadoop/naiad
%         \item the state of current racks allow for comprable performance with DSM
%         \item (small contribution) These system admit failures, but they require expertise to reason about. Applications should be reasonably fault tolerant to prevent heartache, and confusion.
% \end{itemize}


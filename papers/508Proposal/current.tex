\section{Current System}
\label{sec:current}
We plan to extend the existing BlueBridge system, a distributed shared memory cluster written in C. In current BlueBridge, clients are exposed to a global single address space, which is shared across all nodes in the system.

Each client operates on its own local in-memory cache, backed by a customized virtual memory layer.
If this local cache is full, the client process will look up the corresponding remote address in the page table and evict the page on a FIFO-basis.
Servers act as simple memory-banks with the only purpose of storing and retrieving pages. They do not provide protection nor do they enforce access policies.
Single-threaded C applications such as grep or wordcount are able to run on the remote memory cache, any paging is done automatically in the BlueBridge virtual memory system.
BlueBridge does not yet support threading or sharing among processes.

A novel aspect of BlueBridge is the addressing mechanism. Remote addresses are represented as 128-bit IPv6-compatible pointers. On the client side, all remote addresses are stored as combination of the virtual pointer provided by the service node and its source IP address. If a client accesses remote memory, it inserts the address into the IP header of the request packet. The switch will route it automatically to the correct server node. Directory and location management is implicit by leveraging the available naming scheme of networks.
BlueBridge is engineered to provide low latency memory access on the order of a few microseconds. Correspondingly, it currently uses its own networking processing pipeline to support the IPv6 addressing scheme and latency requirements.

BlueBridge's design of several clients accessing a static single address space is a suitable foundation to emulate a disaggregated memory architecture. Client threads represent racks of CPUs with only local caches, servers are interpreted as disaggregated and "dumb" memory, and the rack switch serves as the memory manager of the system.
BlueBridge can be modelled as a NUMA system, which may implement virtual address translation, access policy enforcement, request balancing, and cache coherence completely in the network.

In this project, we will expand on several features of this system and explore design possibilities in the space of network-level shared memory. The precise proposals are as follows.

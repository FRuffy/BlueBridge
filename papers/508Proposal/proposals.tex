\section{Proposals}
\label{sec:proposal}
In this section we propose experimental features of a disaggregated
memory system, and extensions to an existing distributed shared memory
system which will allow developers to write more expressive
applications on top of it.


\noindent{\textbf{Disaggregated memory striping:}} In a disaggregated
rack the memory allocated to a single application may be resident on
separate machines. This begs the question, what are the operational
semantics for a failure of part of a memory allocation? A variety of
options are available; The traditional approach would see the
application share the fate of its memory, and simply crash. In
contrast the contents of the memory could be replicated, and in the
face of a failure, a memory replica is transitioned to transparently.
We propose to use an approach already common in disks, mainly the RAID
architecture. 

Our proposal is to implement main memory raiding at the page level.
When an application page fault occurs, the page fault is intercepted
by a raiding manager which implements a given raid algorithm, either
(0, 1 or 6), and stripes the evicted page across memory blades. To
retrieve pages, the page manager requests the striped page from memory
blades and checks for the pages correctness. In the case of a failure,
our proposed system has the same semantics as traditional raid. If a
page is missing, it is reconstructed using parity raid 2-6. Or in the
case of raid 0 the data is lost. In the face of truly lost data (raid
0 or 1) an entire application should share the fate of the memory
blade.

Using main memory raid has multiple attractive properties. As already
noted raid allows for correctness in the face of partial failures. In
addition to failures, bit level memory correctness is also checked
which allows for optimization's to be made at the network later. For
example remote pages need not use TCP checksums. Finally this
architecture has the advantage that memory bandwidth is multiplied
almost linearly by the number of raided memory blades.

To evaluate our disaggregated memory raided system we will implement
popular graph processing algorithms (label propagation, and pagerank)
and measure their performance relative to benchmarks reached by other
frameworks, such as Naiad, and GraphX. Additionally we will measure
the rate of recovery of applications in the face of failures. 


\noindent{\textbf{Multi-Threaded Applications:}} To date our
distributed memory simulator only has support for single threaded
applications, which share no resources. In order to investigate the
appropriate OS semantics for applications written on a disaggregated
architecture memory sharing and multi-threading are necessary. Adding
these additional features requires re-engineering a portion of the
Blue Bridge distributed shared memory system.

In the current architecture a single thread is allocated a local
scratch space, when the scratch space is filled, and an application
page faults, a request for distributed memory is made. The request has
no information about the thread which issued the request or the scope
of its distributed memory access. We propose an extension to the
distributed memory allocation protocol, in which requests are
identified by their memory offset, and the thread which issued the
request. Additionally the system will require extensions to the memory
manager so that multiple local scratch spaces can be maintained
concurrently.

The addition of multi threaded applications raises the problem of
isolation. Currently BlueBridge gives a single thread unlimited access
to a 128 bit address space. In order to implement isolation, a manager
must maintain an offset on which a given thread can operate on all
memory blades. In essence this extension is simplified virtual memory
for a distributed system.

With the addition of isolation comes the need to facilitate sharing.
We propose the use of named buffers at known locations to allow
sharing between isolated threads. Additionally two threads will have
the ability to operate in the same address space for efficient
computation.

\noindent{\textbf{DPDK:}}
\begin{itemize}
\item Real applications will require real benchmarks
\item Faster is always better
\item We can get actual wins using DSM
\end{itemize}

\noindent{\textbf{Switch MMU:}} Access to distributed memory is currently administered by a
user space process collocated with an executing application. This
application is trapped into when page faults occur, and it issues
requests for remote memory. While practical, this approach does not
take advantage of key aspects of disaggregated architecture at the
scale of a rack.

In a pure disaggregated architecture, CPU’s do not maintain a cache of
data, all memory requests are remote. This property causes all memory
access requests to pass through a TOR. This routing property allows
for memory management to be centralized on a TOR.

-This part is weak.


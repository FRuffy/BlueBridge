%%%%%%%%%%%%%%%%%%%%%%%%%%%%%%%%%%%%%%%%%%%%%
\section{Partitioning}
\label{sec:partition}
%%%%%%%%%%%%%%%%%%%%%%%%%%%%%%%%%%%%%%%%%%%%%

Our partitioning algorithm aims to provide an optimal environment
for the parallel processing of graphs. We leverage the METIS
~\cite{Lasalle:2013:metis} library, an MPI implementation for parallel
partitioning of large graphs, to determine an optimal partitioning for the system.
Metis is built in and runs in C, and the corresponding Python binding, PyMetis ~\cite{pymetis}
was used to determine the optimal graph partitioning it is read into memory.

% Include performance of partitioning here?

Naturally, partitioning a graph will cause edges in the graph to cross 
partitions. Applying techniques seen in ~\cite{Tian:2013:thinklikeagraph} and 
~\cite{Chen:2015:powerlyra}, we use a "mirroring" technique for storing edges
between partitions. For each edge that crosses partitions, only vertices on 
two separate partitions will be affected. The partition which contains the 
destination vertex will be considered the master of the given edge. All edges 
which connect to the master vertex from a different partition will create a
"mirror" vertex. These edges are stored in shared memory, such that they can
be written to and read from different machines.

% TODO add diagram


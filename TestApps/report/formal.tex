%%%%%%%%%%%%%%%%%%%%%%%%%%%%%%%%%%%%%%%%%%%%%
\section{Distributed state background}
\label{sec:formal}
%%%%%%%%%%%%%%%%%%%%%%%%%%%%%%%%%%%%%%%%%%%%%

%% Performing analysis on distributed state requires a method for
%% observing it.
%

% JS: I would like to know why I am explained all this. Why is it
% important that I learn all this? What kind of state invariant should
% I imagine?  Perhaps it is simpler to describe consistent cut with
% math, the words confused me. "A cut is consistent if, for all events
% e and e': ( e ∈ C and e ' → e ) ⇒ e ' ∈ C ". is clearer, no?

In this section we overview our model of distributed state and how it
can be observed from the partially ordered logs of an
execution. Appendix~\ref{sec:formal-appendix} contains a more formal
description.

During a system's execution, each instruction is an \emph{event} and an
\emph{event instance} is a reference to a specific \emph{event}. The
\emph{state} of a node at an event instance is the set of values for
all the variables resident in memory.  The state of a node can be
recorded by writing the variable values to a log.  A \emph{state
  transition} is the change to some variable values in response to an
event instance. An execution of a node is a totally ordered sequence
of events corresponding to state transitions.
%
In a message passing system the sending and receiving events allow a
partial ordering of events across nodes. Vector clocks can
establish this ordering~\cite{mattern_vector_clocks_1989}. We will use
\emph{log} to refer to the sequence of node states paired with vector
clocks generated in a single execution of the system.

Most of Dinv's analyses run on a log produced by a system
after it has executed. Determining the combination of states across
the nodes that should be used to detect meaningful distributed
invariants requires additional constraints.
% 
A \emph{\scc} is a consistent cut in which all messages sent up until
that point have also been received (i.e., no messages are in flight). We
also use \scc to denote the corresponding set of system node states.
%
Inferring state invariants over a \scc ensures that no messages in
flight can invalidate an invariant on arrival. The complete set of
\scc{s} which occur during the execution of a system provide a
step-by-step view of the system's global state transitions on which
Dinv performs its analysis.

To infer distributed invariants Dinv aggregates the state of nodes in
a \scc into a combined global state. It does this in several ways,
each of which is a \emph{merging strategy}. These produce alternative
views of a system. We refer to points at which two or more states are
merged by our log merging algorithm as a \emph{distributed program
  point}. %% Section~\ref{sec:log-analysis} describes three approaches
%% for building distributed program points, all of which operate at the
%% granularity of \scc.
Next, we describe \dinv's design.
%

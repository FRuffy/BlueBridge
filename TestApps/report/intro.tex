%%%%%%%%%%%%%%%%%%%%%%%%%%%%%%%%%%%%%%%%%%%%%
\section{Introduction}
\label{sec:intro}
%%%%%%%%%%%%%%%%%%%%%%%%%%%%%%%%%%%%%%%%%%%%%

Big data processing is complicated. Typically scientists, and
experienced programmers alike struggle with managing and configuring
clusters of machines to processes large amounts of data. Frameworks
like Hadoop and Pregal have significantly eased the difficulty of big
data processing, but they remain intimidating for the lay man. We
noticed a trend in non-systems scientists, and researchers, that they
wanted to \textit{"Just write python code that ran on a bunch of
computers"}. For the benefit of science we investigated how to make
this dream a reality.

Making all Python code run on clusters of machines is impractical, and
due to network overhead would lead to sluggish un-optimal code. Instead
we concentrated our efforts on a common, but difficult big data
processing task, graph processing. Many frameworks exist for
distributed graph
processing~\cite{Malewicz:2010:PSL:1807167.1807184,Ching:2015:OTE:2824032.2824077,Kyrola:2012:GLG:2387880.2387884,Low:2012:DGF:2212351.2212354,Xin:2013:GRD:2484425.2484427,Gonzalez:2012:PDG:2387880.2387883},
and many general frameworks exist which are used for graph
processing~\cite{Vavilapalli:2013:AHY:2523616.2523633,Zaharia:2012:RDD:2228298.2228301,Isard:2007:DDD:1272996.1273005,Murray:2013:NTD:2517349.2522738}.
These frameworks vary in their complexity, but none are
\textit{"accessible"} for programmers with no systems experience.

With the exception of~\cite{Kyrola:2012:GLG:2387880.2387884} the
aforementioned systems suffer a common pitfall to accessibility; they
expose the complexity of a distributed message passing system to the
user. Extensive work has been done to hide this complexity in the
abstraction of distributed shared memory (DSM)
~\cite{Keleher:1994:TDS:1267074.1267084,Power:2010:PBF:1924943.1924964,Morin:2004:KDP:1111682.1111729,Haddad:2001:MCL:374794.374800,Huang06vodca:view-oriented}.
The benefits of DSM have been ignored in recent years due its flaws,
mainly fate sharing and sub optimal performance. Dismissing DSM may
well have been a shortsighted mistake. Ultra dense memory, and the
approach of terabit bandwidth within a rack give modern racks the
appearance of single machines, and has lead towards disaggregated
architectures~\cite{facebook-rack,machine,intel-rsa,seamicro,Han:2013:NSR:2535771.2535778}.
Such futuristic systems lend themselves naturally to DSM which
motivates our proposal for a corresponding computation framework.

The largest disadvantage of DSM is performance. Programmers can write
terribly performant programs by failing to reason about the location
of memory, leading to memory thrashing. In computation where a high
degree of consistency between shared resources DSM is the wrong tool
for the job. In contrast when large amounts of computation can be
performed between memory synchronizations DSM provides a simple and
efficient programming model. Graph processing suffers from a lack of
locality. In computations such as PageRank a single iteration may
require edge updates which require the synchronization of every
machine in a cluster. This problem can be largely avoided in practice
by carefully pre-processing graphs into partitions where the minimum
number of graph edges cross machines. The cost of pre-processing a
graph can be large, in some cases the complexity of finding a good
partition is greater than solving the initial problem! Here we
demonstrate that the cost of graph partitioning is worth it for the
benefits that DSM provides.

In this paper we attack the problem of developing a simple and
efficient graph processing interface for DSM. Specifically we make the
following contributions.

\begin{itemize}
        \item A simple graph processing api
        \item A graph partitioning scheme optimal for DSM
        \item An evaluation of processing performance between partitioned DSM processing, and pregal style graph processing
\end{itemize}

The remainder of this paper is organized as follows. In
Section~\ref{sec:related} we over view related work. In
Section~\ref{sec:api} we describe our graph processing API.
Section~\ref{sec:partition} we describe our approach to graph
partitioning.In Section~\ref{sec:eval} we evaluate our framework
against a pregal style \textit{think like a vertex model}.
Section~\ref{sec:discussion} describes our experiences with our
system, and Section~\ref{sec:conclusion} concludes the paper.





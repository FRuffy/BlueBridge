%%%%%%%%%%%%%%%%%%%%%%%%%%%%%%%%%%%%%%%%%%%%%%%%%%%
\begin{abstract}
%%%%%%%%%%%%%%%%%%%%%%%%%%%%%%%%%%%%%%%%%%%%%%%%%%%

The modern internet generates petabytes of data per day. Processing
    vast amounts of data is an increasingly common task both for
    scientists and modestly experienced programmers. Often this data
    is naturally represented as a graph, such as social media networks,
    webpage links or city networks, and requires clusters of machines 
    to process. Concurrent trends in data centre architecture suggest that 
    the rack is the new server, and shared memory is now a feasible 
    interface between collocated rack servers. These trends made us wonder: 
    \textit{How simple can fast graph processing be on a rack of servers?}. 
    
    We investigate high performance single threaded applications, the
    effort required to muti-thread them, and their usefullness as a
    benchmark for scalable applications. In addition we propose a DSM
    framework for seemlessly scaling multithreaded applications to
    clusters of machines.


%%%%%%%%%%%%%%%%%%%%%%%%%%%%%%%%%%%%%%%%%%%%%%%%%%%
\end{abstract}
%%%%%%%%%%%%%%%%%%%%%%%%%%%%%%%%%%%%%%%%%%%%%%%%%%%

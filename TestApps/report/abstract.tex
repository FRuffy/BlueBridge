%%%%%%%%%%%%%%%%%%%%%%%%%%%%%%%%%%%%%%%%%%%%%%%%%%%
\begin{abstract}
%%%%%%%%%%%%%%%%%%%%%%%%%%%%%%%%%%%%%%%%%%%%%%%%%%%

The modern internet generates petabytes of data per day. Processing
    vast amounts of data is an increasingly common task both for
    scientists and modestly experienced programmers. Often this data
    is naturally represented as a graph, such as social media networks,
    webpage links or city networks, and requires clusters of machines 
    to process. Concurrent trends in data centre architecture suggest that 
    the rack is the new server, and shared memory is now a feasible 
    interface between collocated rack servers. These trends made us wonder: 
    \textit{How simple can fast graph processing be on a rack of servers?}. We
    investigated the tradeoffs of the conventional Pregel
    \textit{"think like a vertex"} programming model, and found its
    performance unacceptable. In contrast, we explored the merits of a 
    \textit{"Think like a subgraph"} model, which respects graph locality 
    in common graphs, provides a more holistic programming interface, and runs
    faster! 

%%%%%%%%%%%%%%%%%%%%%%%%%%%%%%%%%%%%%%%%%%%%%%%%%%%
\end{abstract}
%%%%%%%%%%%%%%%%%%%%%%%%%%%%%%%%%%%%%%%%%%%%%%%%%%%

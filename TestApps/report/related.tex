%%%%%%%%%%%%%%%%%%%%%%%%%%%%%%%%%%%%%%%%%%%%%
\section{Related work}
\label{sec:related}

%%%%%%%%%%%%%%%%%%%%%%%%%%%%%%%%%%%%%%%%%%%%%

The concept of using shared memory in parallel processing of graphs has been
explored by Shun, who created Ligra: a graph processing framework for shared
memory. ~\cite{Shun:2013:ligra} Ligra uses a vertex-subset level of consistency,
and therefore fails to reason about the overall structure of the graph. One
could use Shun's system in a fashion similar to BlueBridge by mapping the
partitioning scheme onto the VertexMap design of Ligra. As Ligra processes
vertices, it uses a frontier-based approach, performing vertex-centric
computations in a breadth-first search matter. Ligra does no work in
pre-partitioning the graph since it is not optimized for distributed shared
memory and instead assumes that entire graphs can fit in a shared memory
server. In settings that instead use distributed shared memory, applying
Ligra's computation model would require constant swapping of memory pages,
which would incur large amounts of network latency during computation.

GraphLab  is another system for parallel processing of Graphs, that could also
extend to a shared memory model. ~\cite{Low:2012:DGF:2212351.2212354} GraphLab
allows for a tunable consistency model (vertex-centric, edge-centric or full)
while performing computations and does allow graph partitioning with ParMetis,
the same library used by BlueBridge. However, GraphLab is a system with much
more infrastructure, reading input files from a distributed storage system
such as HDFS and compressing these partitions into binary files called atoms.
GraphLab provides strong consistency guarantees and scales well but contains
far too much overhead for the everyday scientist.

Piccolo is a distributed shared memory system that also performs pre-partitioning
of data with respect to locality. Piccolo also uses a key-value table interface 
for defining access to shared memory. ~\cite{Power:2010:PBF:1924943.1924964} Since 
Piccolo is also designed for DSM, locality is of high importance when accessing
shared table entries, and Piccolo attempts to optimize the collocation of shared 
table entries. Unlike BlueBridge, Piccolo does not use a persistent RDD storage
model, and instead relies on checkpointing of shared tables through a Chandy-Lamport
distributed snapshot algorithm to provide fault tolerance.